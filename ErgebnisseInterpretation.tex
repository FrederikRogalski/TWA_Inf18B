\documentclass{article}
\usepackage[utf8]{inputenc}
\begin{document}
\section{Ergebnisse}
\subsection{Erlernen der Grundsätze wissenschaftlicher Versuche}
Durch den Versuch wurde den Studenten die Grundsätze der wissenschaftlichen Versuchsdurchführung praxisorientiert nähergebracht. Es wurde die Wichtigkeit der Dokumentation verdeutlicht, indem nach dem Versuch abgefragt wurde, welche Teilnehmer des Kurses sich die Messwerte und den Versuchsaufbau über längere Zeit merken können.
\subsection{Verbesserung der Gruppendynamik im Laufe des Experimentes}
Zu Beginn wurden die Studenten, die den Versuch augeführten, noch häufig ausgelacht, wenn sie einen Fehler begangen. Während des fortschreitenden Versuches verbesserte sich jedoch die Einstellung der Kommilitonen und es wurde konstruktive Kritik eingebracht, zuerst zaghaft, doch es war eine stetig wachsende Hilfsbereitschaft zu beobachten. In Folge dessen konnten die Laboranden die Durchführung des Experimentes entweder präziser oder effizienter gestalten. Des weiteren wurde die Stimmung in der Gruppe immer besser. Es kamen erregte Unterhaltungen auf und die im Regelfall wenig interessierten Studierenden nutzten ihre kombinierte Gedächtnisleistung zur Erreichung des Versuchzieles. Außerdem war zu erkennen, dass sich Studenten zusammenfanden, welche zuvor wenig bis keinen Konakt zueinander besaßen.
\subsection{Zusammenarbeit in wissenschaftlichen Projekten}
Laut Definition ist die Teamarbeit "die Zusammenarbeit einer Gruppe von Personen, um ein gemeinsames Ziel zu erreichen"(Links zur Lesbarkeit entfernt). Die Gruppe von Personen bestand in diesem Fall aus dem Kurs, der mit der Ausführung des Versuches betreut wurde. Dem Kurs wurde die Aufgabe erteilt, zu bestimmen, ob es möglich ist, verschiedene Sorten von Schokolade anhand ihrer Dichte zu unterscheiden, d.h. das gemeinsame Ziel der Teamarbeit lag darin, zu einer Aussage über die Möglichkeit der Unterscheidung der Tafeln Schokolade zu kommen. Somit kann man bei diesem Experiment von Teamarbeit sprechen.
\section{Interpretation}
Von der Dozentin Prof. Dr. Ruth Heine wurde ein Experiment gewählt, sodass durch die Bewegung im Klassenraum die vermittelten Arbeitsmethoden besser verstanden und gemerkt werden konnten. Man erkennt auch die Wichtigkeit von Teamarbeit. Ohne diese hätte der Versuch nicht in der Zeit durchgeführt werden können, in der er hier durchgeführt wurde. Auch die Resultate sind durch Innovationen in der Art, wie die Messungen vollbracht wurden, viel präziser geworden. Die Verbesserung der Stimmung hat allgemein zu einer offeneren und dynamischeren Versuchsumgebung geführt, welche wiederum positive Auswirkungen auf die Produktivität der Studenten hatte. Allgemein lässt sich sagen, dass das Thema der Teamarbeit auch in der Wissenschaft ein sehr Wichtiges ist und durch die Kombination verschiedener Tätigkeitsfelder unbegrenzte Möglichkeiten entstehen. Die Durchführung eines solchen Experimentes, während sich die Beteiligten noch kaum kennen, fördert die Teambereitschaft aller aktiven Teilnehmer. Der Versuch ist demnach sehr gut geeignet in der Findungsphase, in der eine neue (Arbeits-)Gruppe zusammesgestellt wird, die dann effizienter als jeder für sich allein Aufgaben bewältigen und Problemstellungen lösen können.
\end{document}