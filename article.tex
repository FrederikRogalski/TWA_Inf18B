\documentclass[12pt]{scrartcl}
\usepackage[ngerman]{babel}
\usepackage[utf8]{inputenc}
\usepackage[T1]{fontenc}
\usepackage{caption}
\usepackage{graphicx}
\usepackage{booktabs}

\title{Unterscheidung von Schokolade anhand ihrer Dichte \LaTeX{} Article}
\author{Frederik Rogalski}
\begin{document}

\maketitle

\section{Abstract}
Durch präzises Messen und durch wissenschaftliche Überlegungen konnten wir die Unterschiede von Schokoladen wissenschaftlich erkennen und beweisen.
\section{Motivation}
Motiviert hat uns die Unterschiedlichkeit die wir in den einzelnen Schokoladen vorfinden. Keine Schokolade ist wie die andere. Jedoch wollten wir dies Wissenschaftlich beweisen.
\section{Material und Methoden}
Materialien: \\
- Schoko \\
- Wasser \\
- Seife \\
- Becher \\
- Waage \\ \\
Formel: $\rho = \frac{m}{V}$
\pagebreak
\section{Ergebnisse}
	\begin{table}[h!]
	  \center
	  \caption{Oreo}
      \label{tab:oreo}	  
      \begin{tabular}{l|r|r|r|r|r}
      \toprule
      Tafel Nr. & 1 & 2 & 3 & 4 & 5\\
      \midrule
      Masse[$g$] & 100 & 101 & 100 & 98 & 100\\
      \midrule
      Verdrängte Wassermasse [$g$] & 93 & 90 & 87 & 74 & 91\\
      \midrule
      Berechnetes Volumen Schokolade [$cm^3$] & 93 & 90 & 87 & 74 & 91\\
      \midrule
      berechnete Dichte [$g/cm^3$] & 1.075 & 1.122 & 1.149 & 1.324 & 1.098 \\
      \midrule
      relative Messungenaugigkeit Dichte & 0.015 & 0.015 & 0.015 & 0.017 & 0.015 \\
      \midrule
      absolute Messungenauigkeit Dichte [$g/cm^3$] & 0.016 & 0.017 & 0.018 & 0.022 & 0.016 \\
      \bottomrule
      \end{tabular}
	\end{table}
\begin{table}[h!]
	  \center
	  \caption{Alpenmilch}
      \label{tab:alpenmilch}	  
      \begin{tabular}{l|r|r|r|r|r}
      \toprule
      Tafel Nr. & 1 & 2 & 3 & 4 & 5\\
	  \midrule
      Masse[$g$] & 98 & 97 & 100 & 101 & 101\\
      \midrule
      Verdrängte Wassermasse [$g$] & 89 & 79 & 93 & 90 & 88\\
      \midrule
      Berechnetes Volumen Schokolade [$cm^3$] & 89 & 79 & 93 & 90 & 88\\
      \midrule
      berechnete Dichte [$g/cm^3$] & 1.101 & 1.227 & 1.076 & 1.122 & 1.147 \\
      \midrule
      relative Messungenaugigkeit Dichte & 0.015 & 0.016 & 0.015 & 0.015 & 0.015 \\
      \midrule
      absolute Messungenauigkeit Dichte [$g/cm^3$] & 0.017 & 0.02 & 0.016 & 0.017 & 0.017 \\
      \bottomrule
      \end{tabular}
	\end{table}

Unterschiedliche Schokoladen haben unterschiedliche Dichten.
\section{Fazit}
Es war eine tolle Erfahrung dieses aufregende Experiment durchzuführen, welches uns vor allem auch als Team näher gebracht hat. Besonders hat mir gefallen, dass wir die leckere Schokolade danach verspeisen durften.
\section{Verweise}
Alle Information stammen von \textbf{Prof. Dr. Schoki} und unterliegen strengstem vertrauen! 
\end{document}