\documentclass[12pt]{scrartcl}
\usepackage[ngerman]{babel}
\usepackage[utf8]{inputenc}
\usepackage[T1]{fontenc}
\usepackage{caption}
\usepackage{hyperref}
\usepackage{graphicx}
\usepackage{booktabs}

\title{Unterscheidung von Schokolade anhand ihrer Dichte \LaTeX{} Article}
\author{Frederik Rogalsk, Nicolas Kenneman, Valentin Aßfalg}
\begin{document}

\maketitle

\section{Abstract}
Durch präzises Messen und durch wissenschaftliche Überlegungen konnten wir die Unterschiede von Schokoladen wissenschaftlich erkennen und beweisen.
\section{Einleitung}
Teambuilding hat viele Vorteile. Es erhöht die Produktivität von Gruppen und stärkt den Zusammenhalt. Es baut Barrieren ab und erschafft Freundschaften. All das war bei den drei neuen Studenten der DHBW Mosbach nicht so. Die drei dualen Studenten Rederik F., Kicolas N. und Aalentin V. trafen sich in ihrem Kursraum für Angewandte Informatik. Durch ihren Studiengang leiden sie an sozialen Inkompetenzen. Das resultierende Schweigen war selbst dem Dozenten Herr Saller unangenehm. Deshalb beschloss er, die Studenten auf ein Seminar für Teambuilding zu schicken. 

Neben bei wollte er die Studie Aristoteles\textsuperscript {[1]} überprüfen und machte sich während des Teambuildings viele Notizen über das Verhalten der drei Studenten. Da die drei Studenten allerdings Wissenschaftler sind, dachte sich Herr Saller eine Aufgabe aus. Diese Aufgabe war es Schokolade anhand ihrer Dichte zu bestimmen.

\section{Versuch}
Der Versuch bestand aus 10 Tafeln Schokolade, wobei 5 Tafeln Alpenmilch-Schokolade waren und die restlichen aus Oreo-Schokolade bestanden. Zuerst wurden die Tafeln Schokolade verwogen und die Daten in eine Tabelle geschrieben. Anschließend musste man noch das Volumen bestimmen aufgrund der Unebenheit von Schokolade beschloss man dies, mit dem Verdrängungsvolumen herauszufinden. Man stellte nun einen leeren Behälter auf die Waage und tarierte diese. Daraufhin stellte man einen weiteren Behälter, der randvoll mit Wasser gefüllt war, in den ersten Behälter. In den gefüllten Behälter wurde Schokolade gelegt, und das verdrängte Wasser floss in den Auffangbehälter. Anschließend wurde der gefüllte Behälter samt Schokolade entnommen. Übrig bleibt das Gewicht des verdrängten Wassers. Sowohl Rederik F. als auch Kicolas N. ließen eine der Tafeln Schokolade fallen. Darüber konnte Aalentin gut lachen, bis ihm der Behälter mit Wasser auf den Boden fiel. Deswegen wurde er von den anderen beiden Ausgelacht. Allerdings verging allen drei das Lachen, als Sie bemerkten, das sie aufgrund der Oberflächen Spannung von Wasser immense Messfehler hatten. Man behob diesen Fehler durch einen Tropfen Seife im Wasser. Nachdem alles erneut Vermessen wurde entstand, eine Tabelle:
	\begin{table}[h!]
	  \center
	  \caption{Oreo}
      \label{tab:oreo}	  
      \begin{tabular}{l|r|r|r|r|r}
      \toprule
      Tafel Nr. & 1 & 2 & 3 & 4 & 5\\
      \midrule
      Masse[$g$] & 100 & 101 & 100 & 98 & 100\\
      \midrule
      Verdrängte Wassermasse [$g$] & 93 & 90 & 87 & 74 & 91\\
      \midrule
      Berechnetes Volumen Schokolade [$cm^3$] & 93 & 90 & 87 & 74 & 91\\
      \midrule
      berechnete Dichte [$g/cm^3$] & 1.075 & 1.122 & 1.149 & 1.324 & 1.098 \\
      \midrule
      relative Messungenaugigkeit Dichte & 0.015 & 0.015 & 0.015 & 0.017 & 0.015 \\
      \midrule
      absolute Messungenauigkeit Dichte [$g/cm^3$] & 0.016 & 0.017 & 0.018 & 0.022 & 0.016 \\
      \bottomrule
      \end{tabular}
	\end{table}
\begin{table}[h!]
	  \center
	  \caption{Alpenmilch}
      \label{tab:alpenmilch}	  
      \begin{tabular}{l|r|r|r|r|r}
      \toprule
      Tafel Nr. & 1 & 2 & 3 & 4 & 5\\
	  \midrule
      Masse[$g$] & 98 & 97 & 100 & 101 & 101\\
      \midrule
      Verdrängte Wassermasse [$g$] & 89 & 79 & 93 & 90 & 88\\
      \midrule
      Berechnetes Volumen Schokolade [$cm^3$] & 89 & 79 & 93 & 90 & 88\\
      \midrule
      berechnete Dichte [$g/cm^3$] & 1.101 & 1.227 & 1.076 & 1.122 & 1.147 \\
      \midrule
      relative Messungenaugigkeit Dichte & 0.015 & 0.016 & 0.015 & 0.015 & 0.015 \\
      \midrule
      absolute Messungenauigkeit Dichte [$g/cm^3$] & 0.017 & 0.02 & 0.016 & 0.017 & 0.017 \\
      \bottomrule
      \end{tabular}
	\end{table}
Nachdem die Ergebnisse standen, waren alle erschöpft und beschlossen, als Belohnung für getane Arbeit, die Schokolade zu verspeisen. Man erzählte sich Geschichten aus der Schulzeit und lachte gemeinsam.
\section{Motivation}
Motiviert hat uns die Unterschiedlichkeit die wir in den einzelnen Schokoladen vorfinden. Keine Schokolade ist wie die andere. Jedoch wollten wir dies Wissenschaftlich beweisen.
\section{Material und Methoden}
Materialien: \\
- Schoko \\
- Wasser \\
- Seife \\
- Becher \\
- Waage \\ \\
Formel: $\rho = \frac{m}{V}$
\pagebreak

Unterschiedliche Schokoladen haben unterschiedliche Dichten.
\section{Ergebnisse}
\subsection{Erlernen der Grundsätze wissenschaftlicher Versuche}
Durch den Versuch wurde den Studenten die Grundsätze der wissenschaftlichen Versuchsdurchführung praxisorientiert nähergebracht. Es wurde die Wichtigkeit der Dokumentation verdeutlicht, indem nach dem Versuch abgefragt wurde, welche Teilnehmer des Kurses sich die Messwerte und den Versuchsaufbau über längere Zeit merken können.
\subsection{Verbesserung der Gruppendynamik im Laufe des Experimentes}
Zu Beginn wurden die Studenten, die den Versuch augeführten, noch häufig ausgelacht, wenn sie einen Fehler begangen. Während des fortschreitenden Versuches verbesserte sich jedoch die Einstellung der Kommilitonen und es wurde konstruktive Kritik eingebracht, zuerst zaghaft, doch es war eine stetig wachsende Hilfsbereitschaft zu beobachten. In Folge dessen konnten die Laboranden die Durchführung des Experimentes entweder präziser oder effizienter gestalten. Des weiteren wurde die Stimmung in der Gruppe immer besser. Es kamen erregte Unterhaltungen auf und die im Regelfall wenig interessierten Studierenden nutzten ihre kombinierte Gedächtnisleistung zur Erreichung des Versuchzieles. Außerdem war zu erkennen, dass sich Studenten zusammenfanden, welche zuvor wenig bis keinen Konakt zueinander besaßen.
\subsection{Zusammenarbeit in wissenschaftlichen Projekten}
Laut Definition ist die Teamarbeit "die Zusammenarbeit einer Gruppe von Personen, um ein gemeinsames Ziel zu erreichen"(Links zur Lesbarkeit entfernt). Die Gruppe von Personen bestand in diesem Fall aus dem Kurs, der mit der Ausführung des Versuches betreut wurde. Dem Kurs wurde die Aufgabe erteilt, zu bestimmen, ob es möglich ist, verschiedene Sorten von Schokolade anhand ihrer Dichte zu unterscheiden, d.h. das gemeinsame Ziel der Teamarbeit lag darin, zu einer Aussage über die Möglichkeit der Unterscheidung der Tafeln Schokolade zu kommen. Somit kann man bei diesem Experiment von Teamarbeit sprechen.
\section{Interpretation}
Von der Dozentin Prof. Dr. Ruth Heine wurde ein Experiment gewählt, sodass durch die Bewegung im Klassenraum die vermittelten Arbeitsmethoden besser verstanden und gemerkt werden konnten (Katja M. Mayer et al.: "Visual and motor cortices differentially support the translation of foreign language words", in: \textit{Current Biology}, Jahrgang 2015, Heft 4, S.530-535). Man erkennt auch die Wichtigkeit von Teamarbeit. Ohne diese hätte der Versuch nicht in der Zeit durchgeführt werden können, in der er hier durchgeführt wurde. Auch die Resultate sind durch Innovationen in der Art, wie die Messungen vollbracht wurden, viel präziser geworden. Die Verbesserung der Stimmung hat allgemein zu einer offeneren und dynamischeren Versuchsumgebung geführt, welche wiederum positive Auswirkungen auf die Produktivität der Studenten hatte. Allgemein lässt sich sagen, dass das Thema der Teamarbeit auch in der Wissenschaft ein sehr Wichtiges ist und durch die Kombination verschiedener Tätigkeitsfelder unbegrenzte Möglichkeiten entstehen. Die Durchführung eines solchen Experimentes, während sich die Beteiligten noch kaum kennen, fördert die Teambereitschaft aller aktiven Teilnehmer. Der Versuch ist demnach sehr gut geeignet in der Findungsphase, in der eine neue (Arbeits-)Gruppe zusammesgestellt wird, die dann effizienter als jeder für sich allein Aufgaben bewältigen und Problemstellungen lösen können.
\section{Fazit}
Es war eine tolle Erfahrung dieses aufregende Experiment durchzuführen, welches uns vor allem auch als Team näher gebracht hat. Besonders hat mir gefallen, dass wir die leckere Schokolade danach verspeisen durften.
\section{Verweise}
Alle Information stammen von \textbf{Prof. Dr. Schoki} und unterliegen strengstem vertrauen! 
\section{Einzelnachweise}
\begin{enumerate}
\item Google LLC, \href {https://entwickler.de/online/agile/project-aristotle-google-teameffektivitaet-297598.html}{Studie Aristoteles} , USA 2016
\end{enumerate}
\end{document}
