\documentclass[12pt]{scrartcl}
\usepackage[ngerman]{babel}
\usepackage[utf8]{inputenc}
\usepackage[T1]{fontenc}
\usepackage{caption}
\usepackage{hyperref}
\usepackage{graphicx}
\usepackage{booktabs}

\title{Teambuilding anhand eines physikalischen Experimentes}
\author{Frederik Rogalski, Nicolas Kenneman, Valentin Aßfalg}
\begin{document}

\maketitle

\section{Abstract}
When people get to know each other for the first time, they often feel insecure. It takes some time before they feel comfortable and get efficient. This process of teambuilding often takes much time. This paper provides a Method to speed up this process. It discusses if performing a physical Experiment is an eligible Method for teambuilding. The primary target of this paper are people that want to find a way with which they can speed up the process of teambuilding.

\section{Einleitung}
Teambuilding hat viele Vorteile. Es erhöht die Produktivität von Gruppen und stärkt den Zusammenhalt. Es baut Barrieren ab und erschafft Freundschaften. All das war bei den drei neuen Studenten der DHBW Mosbach nicht so. Die drei dualen Studenten Rederik F., Kicolas N. und Aalentin V. trafen sich in ihrem Kursraum für Angewandte Informatik. Durch ihren Studiengang leiden sie an sozialen Inkompetenzen. Das resultierende Schweigen war selbst der Dozentin Prof. Dr. Ruth Heine unangenehm. Deshalb beschloss sie, die Studenten näher zusammen zu bringen. 

Nebenbei wollte sie die Studie Aristoteles \cite{Aristoteles} überprüfen. Dort wurde die Aussage gemacht, dass ein gutes Team nur von der Stimmung und dem Verhalten untereinander abhängt. Da Dozentin Prof. Dr. Ruth Heine sich dies schlecht vorstellen konnte machte sie sich während des Teambuildings viele Notizen über das Verhalten der Probanden. Da es sich allerdings um Wissenschaftler handelt, dachte Sie sich eine Aufgabe aus. Sie sollten Schokolade anhand ihrer Dichte zu bestimmen.

\section{Versuch}
Der Versuch bestand aus 10 Tafeln Schokolade, wobei 5 Tafeln Alpenmilch-Schokolade waren und die restlichen aus Oreo-Schokolade bestanden. Zuerst wurden die Tafeln Schokolade verwogen und die Daten in eine Tabelle geschrieben. Anschließend musste man noch das Volumen bestimmen. Aufgrund der Unebenheit von Schokolade beschloss man, dies mit dem Verdrängungsvolumen herauszufinden. Man stellte nun einen leeren Behälter auf die Waage und tarierte diese. Daraufhin stellte man einen weiteren Behälter, der randvoll mit Wasser gefüllt war, in den ersten Behälter. In den gefüllten Behälter wurde Schokolade gelegt, und das verdrängte Wasser floss in den Auffangbehälter. Anschließend wurde der gefüllte Behälter samt Schokolade entnommen. Übrig bleibt das Gewicht des übergelaufenen Wassers. Sowohl Rederik F. als auch Kicolas N. ließen eine der Tafeln Schokolade fallen. Darüber konnte Aalentin gut lachen, bis ihm der Behälter mit Wasser auf den Boden fiel. Deswegen wurde er von den anderen beiden ausgelacht. Allerdings verging allen drei das Lachen, als Sie bemerkten, das sie aufgrund der Oberflächen Spannung von Wasser immense Messfehler hatten. Man behob diesen Fehler durch einen Tropfen Seife im Wasser. Nachdem alles erneut Vermessen wurde entstand, eine Tabelle:
	\begin{table}[h!]
	  \caption{Oreo}
      \label{tab:oreo}	  
      \begin{tabular}{l|r|r|r|r|r}
      \toprule
      Tafel Nr. & 1 & 2 & 3 & 4 & 5\\
      \midrule
      Masse[$g$] & 100 & 101 & 100 & 98 & 100\\
      \midrule
      verdrängte Wassermasse [$g$] & 93 & 90 & 87 & 74 & 91\\
      \midrule
      berechnetes Volumen Schokolade [$cm^3$] & 93 & 90 & 87 & 74 & 91\\
      \midrule
      berechnete Dichte [$g/cm^3$] & 1.075 & 1.122 & 1.149 & 1.324 & 1.098 \\
      \bottomrule
      \end{tabular}
	\end{table}
\begin{table}[h!]
	  \caption{Alpenmilch}
      \label{tab:alpenmilch}	  
      \begin{tabular}{l|r|r|r|r|r}
      \toprule
      Tafel Nr. & 1 & 2 & 3 & 4 & 5\\
      \midrule
      Masse[$g$] & 98 & 97 & 100 & 101 & 101\\
      \midrule
      verdrängte Wassermasse [$g$] & 89 & 79 & 93 & 90 & 88\\
      \midrule
      berechnetes Volumen Schokolade [$cm^3$] & 89 & 79 & 93 & 90 & 88\\
      \midrule
      berechnete Dichte [$g/cm^3$] & 1.101 & 1.227 & 1.076 & 1.122 & 1.147 \\
      \bottomrule
      \end{tabular}
	\end{table}

In der Tabelle sieht man in den ersten beiden Zeilen die Messwerte. Die gemessene Masse der Tafel Schokolade und die dazugehörige Wassermasse, die verdrängt wurde. Das berechnete Volumen der Schokolade entspricht der Masse, da ein Gramm Wasser exakt einem $cm^3$ entspricht. Die Dichte ergibt sich durch die Teilung von Masse durch Volumen. 

Nachdem die Ergebnisse standen, beschloss man, die Schokolade zu verspeisen. Man erzählte sich Geschichten aus der Schulzeit und lachte gemeinsam.

\section{Ergebnisse}
\subsection{Erlernen der Grundsätze wissenschaftlicher Versuche}
Durch den Versuch wurde den Studenten die Grundsätze der wissenschaftlichen Versuchsdurchführung praxisorientiert näher gebracht. Es wurde die Wichtigkeit der Dokumentation verdeutlicht, indem nach dem Versuch abgefragt wurde, welche Teilnehmer des Kurses sich die Messwerte und den Versuchsaufbau über längere Zeit merken können.
\subsection{Verbesserung der Gruppendynamik im Laufe des Experimentes}
Zu Beginn wurden die Studenten, die den Versuch durchführten, noch häufig ausgelacht, wenn sie einen Fehler begangen. Während des fortschreitenden Versuches verbesserte sich jedoch die Einstellung der Kommilitonen und es wurde konstruktive Kritik eingebracht, zuerst zaghaft, doch es war eine stetig wachsende Hilfsbereitschaft zu beobachten. In Folge dessen konnten die Laboranden die Durchführung des Experimentes entweder präziser oder effizienter gestalten. Des weiteren wurde die Stimmung in der Gruppe immer besser. Es kamen erregte Unterhaltungen auf und die im Regelfall wenig interessierten Studierenden nutzten ihre kombinierte Gedächtnisleistung zur Erreichung des Versuchszieles. Außerdem war zu erkennen, dass sich Studenten zusammenfanden, welche zuvor wenig bis keinen Kontakt zueinander besaßen.
\subsection{Zusammenarbeit in wissenschaftlichen Projekten}
Laut Definition ist die Teamarbeit \glqq die Zusammenarbeit einer Gruppe von Personen, um ein gemeinsames Ziel zu erreichen\grqq \cite{Ergebnisse1}(Links zur Lesbarkeit entfernt). Die Gruppe von Personen bestand in diesem Fall aus dem Kurs, der mit der Ausführung des Versuches betreut wurde. Dem Kurs wurde die Aufgabe erteilt, zu bestimmen, ob es möglich ist, verschiedene Sorten von Schokolade anhand ihrer Dichte zu unterscheiden, d.h. das gemeinsame Ziel der Teamarbeit lag darin, zu einer Aussage über die Möglichkeit der Unterscheidung der Tafeln Schokolade zu kommen. Somit kann man bei diesem Experiment von Teamarbeit sprechen.
\section{Interpretation}
Von der Dozentin Prof. Dr. Ruth Heine wurde ein Experiment gewählt, sodass durch die Bewegung im Klassenraum die vermittelten Arbeitsmethoden besser verstanden und gemerkt werden konnten\cite {Interpretation1}. Man erkennt auch die Wichtigkeit von Teamarbeit. Ohne diese hätte der Versuch nicht in der Zeit durchgeführt werden können, in der er hier durchgeführt wurde. Auch die Resultate sind durch Innovationen in der Art, wie die Messungen vollbracht wurden, viel präziser geworden. Die Verbesserung der Stimmung hat allgemein zu einer offeneren und dynamischeren Versuchsumgebung geführt, welche wiederum positive Auswirkungen auf die Produktivität der Studenten hatte. Allgemein lässt sich sagen, dass das Thema der Teamarbeit auch in der Wissenschaft ein sehr Wichtiges ist und durch die Kombination verschiedener Tätigkeitsfelder unbegrenzte Möglichkeiten entstehen. Die Durchführung eines solchen Experimentes, während sich die Beteiligten noch kaum kennen, fördert die Teambereitschaft aller aktiven Teilnehmer. Der Versuch ist demnach sehr gut geeignet kurz nachdem eine neue (Arbeits-)Gruppe zusammesgestellt worden ist, die dann effizienter als jeder für sich allein Aufgaben bewältigen und Problemstellungen lösen kann, da sie in Zusammenarbeit geschult ist.
\section{Fazit}
Bei der suche nach neuen Möglichkeiten Teambuilding zu betreiben, zeigt sich die Durchführung eines Wissenschaftlichen Experimentes als eine neue Methode. Durch die fast unbegrenzte Anzahl an Experimenten kann für schier jede Gruppe die passende Methode gefunden werden. Auch während der Durchführung eines Experimentes treten häufig Probleme auf, welche gemeinsam im Team gelößt werden müssen. Da diese Aspekte zu den Grundaspekten des Teambuildings gehören, zeigt sich das solche Experimente eine Bereicherung für eine Gruppe sein können bzw. sind. In Zukunft könnte diese Methode häufiger eingesetzt werden, z.~B. in der ersten Studienwoche, damit sich die Studierenden schnell zusammenfinden können und eventuell Arbeitsgemeinschaften oder ähnliches gründen.

\renewcommand{\bibname}{Literaturverzeichnis}
\begin{thebibliography}{1}
  \bibitem{Aristoteles} Project Aristotle – Googles Weg zu mehr Teameffektivität {\em https://entwickler.de/online/agile/project-aristotle-google-teameffektivitaet-297598.html}, Abfrage vom 24.03.2019
  \bibitem{Ergebnisse1} Teamarbeit Definition {\em https://www.onpulson.de/lexikon/teamarbeit/}, Abfrage vom 24.03.2019
  \bibitem{Interpretation1} Katja M. Mayer et al.: \glqq Visual and motor cortices differentially support the translation of foreign language words\grqq, in: \textit{Current Biology}, Jahrgang 2015, Heft 4, S.530-535
\end{thebibliography}
\end{document}
